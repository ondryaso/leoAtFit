\documentclass[12pt,a4paper]{article}
\usepackage[utf8]{inputenc}
\usepackage[czech]{babel}
\usepackage[T1]{fontenc}
\usepackage{graphicx}
\usepackage[left=2cm,right=2cm,top=2cm,bottom=2cm]{geometry}
\usepackage{mathtools}
\usepackage{newtxtext,newtxmath}
\author{Ondřej Ondryáš}
\title{Důkaz vlastnosti determinantů}
\date{2. listopadu 2019}

\begin{document}
\maketitle

Předpokládejme, že:
\[ \det{A \cdot B} = \det{A} \cdot \det{B} \]

Uvažujme, že matice $A$ a $B$ jsou horní trojúhelníkové:
\[
 A = \begin{pmatrix}
  x_{1,1} & x_{1,2} & x_{1,3} \\
  0 & x_{2,2} & x_{2,3} \\
  0 & 0 & x_{3,3}
 \end{pmatrix},\;
 B = \begin{pmatrix}
  y_{1,1} & y_{1,2} & y_{1,3} \\
  0 & y_{2,2} & y_{2,3} \\
  0 & 0 & y_{3,3}
 \end{pmatrix}
\]

Jejich determinanty pak jsou jednoduše:
\begin{align*}
	\det{A} & = x_{1,1} \cdot x_{2,2} \cdot x_{3,3} \\
	\det{B} & = y_{1,1} \cdot y_{2,2} \cdot y_{3,3}
\end{align*}

Součin těchto matic označme $C$:
\[
 C = A \cdot B = \begin{pmatrix}
 	x_{1,1}\cdot y_{1,1} & x_{1,1}\cdot y_{1,2} + x_{1,2}\cdot y_{2,2} & x_{1,1}\cdot y_{1,3} + x_{1,2}\cdot y_{2,3} + x_{1,3}\cdot y_{33} \\
 	0 & x_{2,2}\cdot y_{2,2} & x_{2,2}\cdot y_{2,3} +x_{2,3}\cdot y_{3,3} \\
 	0 & 0 & x_{3,3}\cdot y_{3,3}
 \end{pmatrix}
\]

Matice $C$ je tedy opět horní trojúhelníková. Pro její determinant tak platí:
\[
 \det{C} = x_{1,1}\cdot y_{1,1}\cdot x_{2,2}\cdot y_{2,2}\cdot x_{3,3}\cdot y_{3,3} = \det{A}\cdot \det{B}
\]

Aby mělo tvrzení obecnou platnost, musíme dokázat, že každou čtvercovou matici můžeme převést na horní trojúhelníkovou matici při zachování stejného determinantu a že součinem dvou čtvercových horních trojúhelníkových matic bude opět horní trojúhelníková matice s prvky $a_{i,i}\cdot b_{i,i}$ na hlavní diagonále.

Determinant matice se nezmění, když k jednomu z jejích řádků přičteme $k$-násobek jiného řádku. Úpravami matice tak můžeme vytvořit horní trojúhelníkovou matici se stejným determinantem. (Pokud by neexistovaly vhodné úpravy, aby $a$ a $ea-bd$ nebyly nenulové, determinant můžeme považovat za nulový.)
\renewcommand{\arraystretch}{1.25}
\[
 \begin{pmatrix}
  a & b & c \\
  d & e & f \\
  g & h & i
 \end{pmatrix} \sim
 \begin{pmatrix}
  a & b & c \\
  0 & e - \frac{bd}{a} & f - \frac{cd}{a} \\
  0 & h - \frac{bg}{a} & i - \frac{cg}{a}
 \end{pmatrix} \sim
 \begin{pmatrix}
  a & b & c \\
  0 & e - \frac{bd}{a} & f - \frac{cd}{a} \\
  0 & 0 & \frac{1}{a}(ia-cg-\frac{(fa-cd)(ha-bg)}{ea-bd})
 \end{pmatrix}
\]
\renewcommand{\arraystretch}{1}

Součinem čtvercových matic \(A = (a_{i,j})_{n,n}, B = (b_{i,j})_{n,n}\) je matice \(C = (c_{i,j})_{n,n}\), kde
\[
 c_{i,j} = a_{i,1}\cdot b_{1,j} + a_{i,2}\cdot b_{2,j} + \cdots + a_{i,n}\cdot b_{n,j} = \sum_{r=1}^n a_{i,r}\cdot b_{r,j}
\]

V předpokladu násobíme pouze horní trojúhelníkové matice, pro které platí, že
\[
 i > j \Rightarrow a_{i,j} = 0
\]
Všechny prvky $c_{i,j}$, kde $i > j$, pak budou nulové, protože v každém z prvků sumy bude buď $i > r$, nebo $r > j$, a jedná se tak také o horní trojúhelníkovou matici.

Pro prvky na hlavní diagonále $c_{i,i}$ platí:
\[
 c_{i,i} = \sum_{r=1}^n a_{i,r}\cdot b_{r,i}
\]
Jediný nenulový prvek této sumy bude $a_{i,i}\cdot b_{i,i}$, protože pro všechny ostatní je buď $i > r$, nebo $r > j$, jeden z prvků je tak nulový. Hlavní diagonálu této matice pak budou tvořit právě prvky $a_{i,i}\cdot b_{i,i}$, a determinant tak můžeme psát jako
\[
 \det{C} = (a_{1,1}\cdot b_{1,1})\cdot (a_{2,2}\cdot b_{2,2})\cdot (\cdots) \cdot (a_{n,n}\cdot b_{n,n})
\]

Jakékoliv dvě čtvercové matice stejného typu $A, B$ tak můžeme převést na horní trojúhelníkové matice se stejným determinantem $A', B'$, přičemž
\begin{multline*}
 \det{A\cdot B} = \det{A' \cdot B'} = (a'_{1,1}\cdot b'_{1,1})\cdot (a'_{2,2}\cdot b'_{2,2})\cdot (\cdots) \cdot (a'_{n,n}\cdot b'_{n,n}) \\= (a'_{1,1}\cdot a'_{2,2}\cdot \cdots \cdot a'_{n,n})\cdot (b'_{1,1}\cdot b'_{2,2}\cdot \cdots \cdot b'_{n,n}) = \det{A'}\cdot \det{B'} = \det{A}\cdot \det{B}\;\blacksquare
\end{multline*}
\end{document}