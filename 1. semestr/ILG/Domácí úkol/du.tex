\documentclass[12pt,a4paper]{article}
\usepackage[utf8]{inputenc}
\usepackage{amsmath}
\usepackage[czech]{babel}
\usepackage[T1]{fontenc}
\usepackage{graphicx}
\usepackage[left=2cm,right=2cm,top=2cm,bottom=2cm]{geometry}
\usepackage{mathtools}
\usepackage{gauss}
\usepackage{verbatim}
\usepackage{listings}

\usepackage{amssymb}
\usepackage{systeme}

\usepackage{color}
\definecolor{bluekeywords}{rgb}{0.13,0.13,1}
\definecolor{greencomments}{rgb}{0,0.5,0}
\definecolor{redstrings}{rgb}{0.9,0,0}

\sysdelim..

\usepackage{array}
\allowdisplaybreaks

\makeatletter
\newcounter{elimination@steps}
\newcolumntype{R}[1]{>{\raggedleft\arraybackslash$}p{#1}<{$}}
\def\elimination@num@rights{}
\def\elimination@num@variables{}
\newenvironment{elimination}[3][0]
{
    \setcounter{elimination@steps}{0}
    \def\elimination@num@rights{#1}
    \def\elimination@num@variables{#2}
    \renewcommand{\arraystretch}{#3}
    \start@align\@ne\st@rredtrue\m@ne
}
{
    \endalign
    \ignorespacesafterend
}
\newcommand{\eliminationstep}[4]
{
    \ifnum\value{elimination@steps}>0\\\sim\quad\fi
    \left(
        \ifnum\elimination@num@rights>0
            \begin{array}
            {@{}*{\elimination@num@variables}{R{#1}}
            |@{}*{\elimination@num@rights}{R{#2}}}
        \else
            \begin{array}
            {@{}*{\elimination@num@variables}{R{\elimination@col@width}}}
        \fi
            #3
        \end{array}
    \right)
    & \quad
    \begin{array}{l}
        #4
    \end{array}
    \addtocounter{elimination@steps}{1}
}
\makeatother


%\usepackage{cancel}
\usepackage{newtxtext,newtxmath}



\title{Domácí úloha}

\newcommand{\pageline}{\noindent\makebox[\linewidth]{\rule{\linewidth}{0.4pt}}\vspace{5pt}}
\setlength{\parskip}{0.75em}



\setlength\abovedisplayshortskip{10pt}
\setlength\abovedisplayskip{10pt}
\setlength\belowdisplayshortskip{10pt}
\setlength\belowdisplayskip{10pt}

%===============================================================================
\graphicspath{{fig/}}

%pagestyle settings
\usepackage{fancyhdr}
\usepackage[export]{adjustbox}


\pagestyle{fancy}
\fancyhf{}
\rfoot{Brno, 10. prosince 2019}
\lhead{Lineární algebra}
\lfoot{2. skupina}
\rhead{Domácí úloha}

\cfoot{\thepage}

% hlavička 

\title{\includegraphics[scale=0.1,keepaspectratio]{fig/logo_cz.png}\\ \huge{Lineární algebra} \\ 2019/2020}
\date{}


\begin{document}
	\maketitle
\thispagestyle{empty}

\begin{center}
    \huge Domácí úloha
\end{center}
\begin{center}
    \huge \text{2. skupina}
\end{center}
\begin{center}
David Mihola  |  xmihol00 \\
Ondřej Ondryáš  |  xondry02 \\
David Chocholatý  |  xchoch08 \\
František Nečas  |  xnecas27 \\
Lukáš Foltyn  |  xfolty17
\end{center}








\newpage
\section*{Příklad 1}

Na množině reálných čísel řešte soustavu rovnic s parametrem $a$.

\[
\systeme[xyz]{
ax + y + z = 1-a,
x + ay + z = 0,
x + y + az = 0
}
\] 
\pageline \newline
Soustavu rovnic upravíme pomocí Gaussovy eliminace.

\begin{elimination}[3]{3}{1.1}
    \eliminationstep{1.5em}{3em}
    {
        a & 1 & 1 & 1 - a \\
        1 & a & 1 & 0 \\
        1 & 1 & a & 0
    }
    {
        \\
        \\
    }
    \eliminationstep{3em}{3em}
    {
        1 & 1 & a & 0 \\
        0 & a-1 & 1-a & 0 \\
        0 & 1-a & 1-a^2 & 1-a
    }
    {
        III\\
        II - III\\
        I - aIII
    }
    \eliminationstep{4.1em}{3em}
    {
        1 & 1 & a & 0 \\
        0 & a-1 & 1-a & 0 \\
        0 & 0 & a^2+a - 2 & a - 1
    }
    {
        I\\
        II\\
        -III - II
    }
\end{elimination}

Eliminovanou soustavu vyřešíme.

\begin{align*}
    (a^2 + a - 2)z = a - 1 \\
    z = \frac{a-1}{(a+2)(a-1)} = \frac{1}{a+2} \tag*{pro $a \notin  \{1, -2\}$}\\
\end{align*}
\begin{align*}
    (a-1)y + \frac{1-a}{a+2} = 0\\
    y = -\frac{1-a}{(a+2)(a-1)} = \frac{1}{a+2} \tag*{pro $a \neq -2$}
\end{align*}
\begin{align*}
    x + \frac{1}{a+2} + \frac{a}{a+2} = 0\\
    x = -\frac{a+1}{a+2}
\end{align*}
\newpage
Dále vyřešíme případy, ve kterých výsledek nelze kvůli dělení nulou spočítat obecně pomocí eliminace, tedy $a = 1$ a $a = -2$.

Pro $a = 1$

\begin{align*}
    (1^{2} + 1 - 2)z &= 1 - 1\\
    0z &= 0 \\
    z &= t \tag*{$t \in \mathbb{R}$}
\end{align*}
\begin{align*}
    0y + 0z &= 0 \\
    y &= s \tag*{$s \in \mathbb{R}$}
\end{align*}
\begin{align*}
    x + y + z = 0\\
    x = -t -s
\end{align*}

Pro $a = -2$

\begin{align*}
    (4 - 2 -2)z &= -3\\
    0z &= -3 \\
    0 &\neq -3
\end{align*}

Pro $a = -2$ neexistuje žádné řešení.

Výsledky můžeme shrnout následovně:
\begin{center}
    \begin{tabular}{c|c}
        $a$ & $\{[x, y, z]\}$\\
        \hline
        $-2$ & $\emptyset$ \\
        $1$ & $\{[-t -s, s, t]| t, s \in \mathbb{R}\}$\\
        $\mathbb{R} \setminus \{1, -2\}$ & $\{[\frac{a+1}{a+2}, \frac{1}{a+2}, \frac{1}{a+2}]\}$
    \end{tabular}
\end{center}



%===============================================================================

\newpage

\section*{Příklad 2}

Určete jenom na základě vlastností determinantů:
\begin{center}
$
\begin{vmatrix}
    z & u & v \\
    v + u & v + z & z + u \\
    1 & 1 & 1
\end{vmatrix}
$
\end{center}

\pageline

Můžeme využívat pouze operací vycházejících z vlastností determinantu, jako jsou operace sčítání a odčítání řádků či sloupců, prohození řádků se změnou znaménka determinantu, násobení konstantou apod.
\begin{enumerate}
    \item Nejprve odečteme druhý sloupec od prvního i třetího sloupce,
    \item první řádek přičteme k druhému řádku,
    \item odečteme $(u + v + z)$-násobek třetího řádku od druhého. Tuto operaci můžeme klidně provést, neboť víme, že ať bude součet $(u + v + z)$ jakýkoliv, nedojde ke změně výsledného determinantu, protože:
    \begin{enumerate}
        \item $(u + v + z) = 0$ ... v determinantu se vyskytuje nulový řádek – determinant je roven 0,
        \item $(u + v + z) \neq 0$ ... provedeme operaci a opět vznikne stejný nulový řádek – determinant je roven 0.
    \end{enumerate}
\end{enumerate}
$$
\begin{gmatrix}[v]
    z & u & v \\
    v + u & v + z & z + u \\
    1 & 1 & 1
\colops
\add[-1]{1}{2}
\add[-1]{1}{0}
\end{gmatrix}
=
\begin{gmatrix}[v]
    z - u & u & v - u \\
    u - z & v + z & u - v \\
    0 & 1 & 0
\rowops
\add[]{0}{1}
\end{gmatrix}
=
\begin{gmatrix}[v]
    z - u & u & v - u \\
    0 & u + v + z & 0 \\
    0 & 1 & 0
\rowops
\add[-(u + v +z)]{2}{1}
\end{gmatrix}
=
$$
$$
=
\begin{gmatrix}[v]
    z - u & u & v - u \\
    0 & 0 & 0 \\
    0 & 1 & 0
\end{gmatrix}
$$


\begin{comment} PUVODNI PODOBA
$$
\begin{vmatrix}
    z & u & v \\
    v + u & v + z & z + u \\
    1 & 1 & 1
\end{vmatrix}
=
\begin{vmatrix}
    z - u & u & v - u \\
    u - z & v + z & u - v \\
    0 & 1 & 0
\end{vmatrix}
=
\begin{vmatrix}
    z - u & u & v - u \\
    0 & u + v + z & 0 \\
    0 & 1 & 0
\end{vmatrix}
=
\begin{vmatrix}
    z - u & u & v - u \\
    0 & 0 & 0 \\
    0 & 1 & 0
\end{vmatrix}
$$
\end{comment}

Druhý řádek determinantu je nulový. Můžeme s jistotou tvrdit, že ať zvolíme libovolná $x, y, z$, výsledek determinantu bude vždy roven nule.

%===============================================================================

\newpage

\section*{Příklad 3}
Nechť $M = {[a, b] \in \mathbb{R}^2; 3\text{ }|\text{ }(a + 2b)}$. Zjistěte, jestli $M$ je podprostor $V_2(\mathbb{R})$. Svou odpověď zdůvodněte.
\pageline

\begin{enumerate}
    \item Podprostor $M$ nesmí být prázdný. Ověříme, že $M \neq \emptyset$.
    V $M$ se tedy musí nacházet minimálně $\overline{0}$. V tom případě by platilo následující:
    $$[a, b] = [0, 0] \Rightarrow 3\text{ }|\text{ }(0 + 2 \cdot 0) \equiv 3\text{ }|\text{ }0$$ Platí. $\overline{0} \in M$. $M$ není prázdný.
    
    \item Náleží-li $M$ dva vektory $\overline{u}$ a $\overline{v}$, musí $M$ náležet i jejich součet.
    \begin{align*}
        &\overline{u}, \overline{v} \in M \Rightarrow \overline{u} + \overline{v} \in M \\
        \overline{u} = [u_1; u_2] \text{ a } \overline{v} = [v_1; v_2] \text{ a jistě víme, že: } \\
        &3\text{ }|\text{ } (u_1 + 2u_2) \text{ a } 3 \text{ }|\text{ } (v_1 + 2v_2) \\
        \text{Součet vektorů je roven:} \\
        &\vec{u} + \vec{v} = [u_1 + v_1; u_2 + v_2] \\
        \text{Aby } \vec{u} + \vec{v} \text{ bylo z } M \text{, muselo by platit: } \\
        &3 \text{ }|\text{ } (u_1 + v_1) + 2(u_2 + v_2) \\
        \text{A protože platí: } \\
        &3 \text{ }|\text{ } (u_1 + 2u_2) +  (v_1 + 2v_2) \\
        \text{Což můžeme upravit na: } \\
        &3 \text{ }|\text{ } (u_1 + v_1) + 2(u_2 + v_2) \\
        \text{Víme tedy, že: } \\ 
        &\overline{u} + \overline{v} \in M\text{.}
    \end{align*}
    Platí.
    
    
    
    \item Jestliže $\overline{u}\in M$ a zároveň existuje $\alpha \in \mathbb{R}$, potom platí: $\alpha \cdot \overline{u} \in M$: 
    \begin{align*}
        &\alpha \cdot \overline{u} = [\alpha \cdot u_1; \alpha \cdot u_2] \\
        &3 \text{ }|\text{ } \alpha \cdot u_1 + \alpha \cdot 2 \cdot u_2 \\
        &3 \text{ }|\text{ } \alpha \cdot (u_1 + 2u_2) \\
        &\text{Zvolíme-li ale například $\alpha = 0,5$ a $\vec{u} = [1; 1]$, dostaneme: } \\
        &\alpha \cdot \overline{u} = [0,5; 0,5] \text{ ale } 3 \nmid (0,5 + 2 \cdot 0,5) 
    \end{align*}
    Neplatí. $\alpha \cdot \overline{u} \notin M \Rightarrow $ Nejedná se o podprostor $V_2(\mathbb{R})$.
\end{enumerate}

%===============================================================================

\newpage
\section*{Příklad 4}
Vektorový podprostor $L \subseteq V_4(\mathbb{R})$ je generován vektory
$$ \overline{a}_1 = [1, - 1, 2, - 2]\text{, }\overline{a}_2 = [1, 1, - 2, 3]\text{, }\overline{a}_3 = [- 1, - 1, 7, - 8]\text{.} $$

Pomocí Gram-Schmidtova ortogonalizačního procesu najděte ortonormální bázi L.

\pageline

\begin{enumerate}
    \item Začneme hledat ortogonální báze $\overline{b}_1$, $\overline{b}_2$ a $\overline{b}_3$ vektorů $\overline{a}_1$, $\overline{a}_2$ a $\overline{a}_3$:
    \begin{align*}
        &\overline{b}_1 = \overline{a}_1 = [1;-1;2;-2] \\
        &\overline{b}_2 = \overline{a}_2 + \beta_{21} \cdot \overline{a}_1 \\
        &\overline{b}_3 = \overline{a}_3 + \beta_{32} \cdot \overline{a}_2 + \beta_{31} \cdot \overline{a}_1 \\
    \end{align*}
\item Vypočteme a dosadíme $\beta_{21}$, $\beta_{32}$ a $\beta_{31}$ podle vzorečku:
$$ \beta_{xy} = - \frac{\overline{a}_x \cdot \overline{b}_y}{\overline{b}_y \cdot \overline{b}_y} $$
    \begin{align*}
        &\beta_{21} = - \frac{\overline{a}_2 \cdot \overline{b}_1}{\overline{b}_1 \cdot \overline{b}_1} = - \frac{[1, 1, -2, 3] \cdot [1;-1;2;-2]}{[1;-1;2;-2] \cdot [1;-1;2;-2]} = \frac{10}{10} = 1  \\
        &\overline{b}_2 = \overline{a}_2 + \beta_{21} \cdot \overline{a}_1 = [1, 1, - 2, 3] + 1 \cdot [1;-1;2;-2] = [2;0;0;1] \\
        &\beta_{32} = - \frac{\overline{a}_3 \cdot \overline{b}_2}{\overline{b}_2 \cdot \overline{b}_2} = - \frac{[- 1, - 1, 7, - 8] \cdot [2;0;0;1]}{[2;0;0;1] \cdot [2;0;0;1]} = - \frac{30}{10} = - 3  \\
        &\beta_{31} = - \frac{\overline{a}_3 \cdot \overline{b}_1}{\overline{b}_1 \cdot \overline{b}_1} = - \frac{[- 1, - 1, 7, - 8] \cdot [1;-1;2;-2]}{[1;-1;2;-2] \cdot [1;-1;2;-2]} = \frac{10}{5} = 2  \\
        &\overline{b}_3 = \overline{a}_3 + \beta_{32} \cdot \overline{a}_2 + \beta_{31} \cdot \overline{a}_1  = [- 1, - 1, 7, - 8] - 3 \cdot [1, 1, - 2, 3] + 2 \cdot [1, - 1, 2, - 2] = [0;2;1;0]
    \end{align*}
    \item Ověříme správnost vypočítaných výsledků pomocí vlastností ortogonálních bází – jsou na sebe navzájem kolmé:
    \begin{align*}
        &\overline{b}_2 \perp \overline{b}_1 \Rightarrow \overline{b}_2 \cdot \overline{b}_1 = 0 \Leftrightarrow [2;0;0;1] \cdot [1;-1;2;-2] = 0 \text{ ... platí} \\
        &\overline{b}_3 \perp \overline{b}_2 \Rightarrow \overline{b}_3 \cdot \overline{b}_2 = 0 \Leftrightarrow [0;2;1;0] \cdot [2;0;0;1] = 0 \text{ ... platí} \\
        &\overline{b}_3 \perp \overline{b}_1 \Rightarrow \overline{b}_3 \cdot \overline{b}_1 = 0 \Leftrightarrow [0;2;1;0] \cdot [1;-1;2;-2] = 0 \text{ ... platí}
    \end{align*}
    \newpage
    \item Dopočítáme ortonormální báze podle vzorečku:
    $$ \overline{h}_x = \frac{\overline{b}_x}{||\overline{b}_x||} $$
    
    \begin{align*}
        &\overline{h}_1 = \frac{\overline{b}_1}{||\overline{b}_1||} = \frac{[1;-1;2;-2]}{\sqrt{1^2 + (-1)^2 + 2^2 + (-2)^2}} = \frac{[1;-1;2;-2]}{\sqrt{10}} = \left[\displaystyle \frac{\sqrt{10}}{10}; - \frac{\sqrt{10}}{10}; \frac{\sqrt{10}}{5}; - \frac{\sqrt{10}}{5} \right ] \\
        &\overline{h}_2 = \frac{\overline{b}_2}{||\overline{b}_2||} = \frac{[2;0;0;1]}{\sqrt{2^2 + 0^2 + 0^2 + 1^2}} = \frac{[1;-1;2;-2]}{\sqrt{5}} = \left[\displaystyle \frac{2\sqrt{5}}{5}; 0; 0; \frac{\sqrt{5}}{5} \right ] \\
        &\overline{h}_3 = \frac{\overline{b}_3}{||\overline{b}_3||} = \frac{[0;2;1;0]}{\sqrt{0^2 + 2^2 + 1^2 + 0^2}} = \frac{[1;-1;2;-2]}{\sqrt{5}} = \left[\displaystyle 0; \frac{2\sqrt{5}}{5}; \frac{\sqrt{5}}{5}; 0 \right ] \\
    \end{align*}
    \item Opět můžeme ověřit správnost vypočítaných výsledků pomocí vlastností ortonormálních bází – jsou na sebe navzájem kolmé:
    \begin{align*}
        &\overline{h}_2 \perp \overline{h}_1 \Rightarrow \overline{h}_2 \cdot \overline{h}_1 = 0 \Leftrightarrow \left[\displaystyle \frac{2\sqrt{5}}{5}; 0; 0; \frac{\sqrt{5}}{5} \right ] \cdot \left[\displaystyle \frac{\sqrt{10}}{10}; - \frac{\sqrt{10}}{10}; \frac{\sqrt{10}}{5}; - \frac{\sqrt{10}}{5} \right ] = 0 \text{ ... platí} \\
        &\overline{h}_3 \perp \overline{h}_2 \Rightarrow \overline{h}_3 \cdot \overline{h}_2 = 0 \Leftrightarrow \left[\displaystyle 0; \frac{2\sqrt{5}}{5}; \frac{\sqrt{5}}{5}; 0 \right ] \cdot \left[\displaystyle \frac{2\sqrt{5}}{5}; 0; 0; \frac{\sqrt{5}}{5} \right ] = 0 \text{ ... platí} \\
        &\overline{h}_3 \perp \overline{h}_1 \Rightarrow \overline{h}_3 \cdot \overline{h}_1 = 0 \Leftrightarrow \left[\displaystyle 0; \frac{2\sqrt{5}}{5}; \frac{\sqrt{5}}{5}; 0 \right ] \cdot \left[\displaystyle \frac{\sqrt{10}}{10}; - \frac{\sqrt{10}}{10}; \frac{\sqrt{10}}{5}; - \frac{\sqrt{10}}{5} \right ] = 0 \text{ ... platí}
    \end{align*}
    
    Nalezli jsme ortonormální báze podprostoru $L$.
\end{enumerate}


%===============================================================================

\newpage
\section*{Příklad 5}





Je dána soustava rovnic
\[
\systeme[xyz]
{
x - 3y + 100z = 123,
200x - 3y + 2z = 765,
x - 500y + 2z = 987
}
\] 

Řešení soustavy najděte s přesností $\epsilon = 0.01$ Gauss-Seidelovou metodou, vyjděte z bodu $(3.5; -2; 1)$. Je-li to potřeba, soustavu nejprve upravte tak, aby byla zaručena konvergence.
\newline \pageline


Konvergenci zaručíme výměnou řádků matice tak, aby byla ostře řádkově dominantní:

\[
\begin{pmatrix} 
    1 & -3 & 100\\
    200 & -3 & 2\\
    1 & -500 & 2
\end{pmatrix} \cdot
\begin{pmatrix} 
    x\\
    y\\
    z
\end{pmatrix} 
=
\begin{pmatrix} 123\\765\\987 \end{pmatrix} 
\]

\[
\begin{pmatrix} 
    200 & -3 & 2\\
    1 & -500 & 2\\
    1 & -3 & 100
\end{pmatrix}
\begin{pmatrix} 
    x\\
    y\\
    z
\end{pmatrix} 
=
\begin{pmatrix} 765\\987\\123 \end{pmatrix} 
\] 


\begin{align*}
    x_{k+1}&=\frac{765 + 3y_k - 2z_k}{200}\\
    y_{k+1}&=\frac{987 - 1x_{k+1} - 2z_k}{-500}\\
    z_{k+1}&=\frac{123 - 1x_{k+1} + 3y_{k+1}}{100}
\end{align*}

$$ x_0 = 3.5, y_0 = -2, z_0 = 1 $$

\begin{align*}
x_{1} &= \frac{765-3\cdot2-2\cdot1}{200} = 3{,}785 \tag*{$\Delta x = 0{,}285$}\\
y_{1} &= \frac{987-1\cdot3{,}785-2\cdot1}{-500} = -1{,}96243 \tag*{$\Delta y = 0{,}03757$}\\
z_{1} &= \frac{123-1\cdot3{,}785-3\cdot1{,}96243}{100} = 1{,}1332771 \tag*{$\Delta z = 0{,}1332771$}\\
\\
x_{2} &= \frac{765-3\cdot1{,}96243-2\cdot1{,}1332771}{200} = 3{,}784230779\,(\approx 3{,}7842) \tag*{$\Delta x \approx 0{,}00077$}\\
y_{2} &= \frac{987-1\cdot3{,}784230779-2\cdot1{,}1332771}{-500} = -1{,}961898430042\,(\approx -1{,}9619) \tag*{$\Delta y \approx 0{,}00053$}\\
z_{2} &= \frac{123-1\cdot3{,}784230779-3\cdot1{,}961898430042}{100} = 1{,}13330073930874\,(\approx 1{,}1333) \tag*{$\Delta z \approx 0{,}00002$}\\
\end{align*}

Řešení $[x_2, y_2, z_2]$ již splňuje zadanou podmínku $\epsilon = 0,01$.

\newpage

\lstset{language=[Sharp]C,
showspaces=false,
showtabs=false,
breaklines=true,
showstringspaces=false,
breakatwhitespace=true,
escapeinside={(*@}{@*)},
commentstyle=\color{greencomments},
keywordstyle=\color{bluekeywords}\bfseries,
stringstyle=\color{redstrings},
basicstyle=\ttfamily
}

\begin{lstlisting}
static Vector GaussSeidel(Matrix system, Vector values, decimal accuracy = 0.01m, Vector startingValues = null,
    int maxCycles = 128, char startChar = 'x')
{
    int a = values.Rows;
    if (a != system.Rows) return null;

    // TODO: convergence
    int i = 0;

    Vector temp = startingValues ?? new Vector(system.Columns);
    if (temp.Rows != system.Columns) return null;
    
    while (i < maxCycles)
    {
        Vector prev = new Vector(temp);
        bool accurateEnough = true;

        for (int v = 0; v < temp.Rows; v++)
        {
            temp[v] = values[v];
            var line = "";
            
            line += $"{(char) (startChar + v)}_{{{i + 1}}} &= \\frac{{{temp[v]}";
            
            for (int c = 0; c < system.Columns; c++)
            {
                if (c == v) continue;
                var val = system[v, c] * temp[c];
                temp[v] -= val;
                
                line += $"{(val > 0 ? "-" : "+")}{Math.Abs(system[v, c])}\\cdot{Math.Abs(temp[c])}";
            }

            temp[v] /= system[v, v];

            line +=
                $"}}{{{system[v, v]}}} = {temp[v]} \\tag*{{$\\Delta {(char) (startChar + v)} = {Math.Abs(prev[v] - temp[v])}$}}\\\\\n";
            
            if(v == temp.Rows - 1) line += "\\\\\n";
            Console.Write(line.Replace(".", "{,}"));

            if (Math.Abs(prev[v] - temp[v]) > accuracy)
            {
                accurateEnough = false;
            }
        }

        if (accurateEnough) break;
        i++;
    }

    if (i == maxCycles) return null;
    return temp;
}
\end{lstlisting}

\end{document}
