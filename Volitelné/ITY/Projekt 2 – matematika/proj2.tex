% ITY – projekt 2
% Ondřej Ondryáš (xondry02), FIT VUT
% Datum: 14. 3. 2020
% Poznámky: – V úplně posledním použitém matematickém prostředí se mé vypracování liší od vzorového dokumentu
%             v šířce mezer za znakem „k“. Zvláštního odsazení ze zadání bych docílil vložením mezery střední
%             délky za znaky „k“ pomocí sekvence \: – mému typografickému vnímání se to ale velmi příčí, a tak
%             jsem si dovolil tuto mezeru vypustit.
%           – V textu je řečeno, že je na titulní straně použito odřádkování se zadanou relativní velikostí,
%             ve vzorovém dokumentu tato odřádkování ale zjevně použita nejsou, proto jsem je také nepoužil.
%             Bylo by jich dosaženo vložením argumentu [0.4em] (resp. [0.3em]) za příslušné příkazy \\.

\documentclass[twocolumn,11pt]{article}

\usepackage[utf8]{inputenc}
\usepackage[IL2]{fontenc}
\usepackage{times}
\usepackage{amsthm}
\usepackage{amsfonts}
\usepackage{amsmath}
\usepackage[a4paper, top=2.5cm, left=1.5cm, textwidth=18cm, textheight=25cm]{geometry}
\usepackage[czech]{babel}
\usepackage{microtype}

\usepackage[pdftex, pdfauthor={Ondřej Ondryáš (xondry02)},
            pdftitle={Typografie a publikování, 2. projekt: Sazba dokumentů a matematických výrazů},
            hidelinks,unicode]{hyperref}

\theoremstyle{plain}
\newtheorem{deft}{Definice}
\newtheorem{lem}{Věta}

\begin{document}

\begin{titlepage}
\begin{center}
\Huge\textsc{Fakulta informačních technologií}\\\textsc{Vysoké učení technické v~Brně}\\
\vspace{\stretch{0.382}}
\LARGE
Typografie a publikování\,--\,2. projekt\\
Sazba dokumentů a matematických výrazů\\
\vspace{\stretch{0.618}}
\end{center}
{\Large 2020 \hfill Ondřej Ondryáš (xondry02)}
\end{titlepage}

\section*{Úvod}
V této úloze si vyzkoušíme sazbu titulní strany, matematických vzorců, prostředí a~dalších textových struktur obvyklých pro technicky zaměřené texty (například rovnice~\eqref{eq:second} nebo Definice~\ref{deft:retezec} na straně~\pageref{deft:retezec}). Pro vytvoření těchto odkazů používáme příkazy \verb|\label|, \verb|\ref| a~\verb|\pageref|.

Na titulní straně je využito sázení nadpisu podle optického středu s~využitím zlatého řezu. Tento postup byl probírán na přednášce. Dále je použito odřádkování se zadanou relativní velikostí 0.4em a 0.3em.

\section{Matematický text}
Nejprve se podíváme na sázení matematických symbolů a~výrazů v plynulém textu včetně sazby definic a~vět s~využitím balíku \verb|amsthm|. Rovněž použijeme poznámku pod čarou s použitím příkazu \verb|\footnote|. Někdy je vhodné použít konstrukci \verb|${}$| nebo \verb|\mbox{}| která říká, že (matematický) text nemá být zalomen. V následující definici je nastavena mezera mezi jednotlivými položkami \verb|\item| na 0.05em.

\begin{deft}
\label{deft:turinguv_stroj}
{\normalfont Turingův stroj} (TS) je definován jako šestice tvaru $M = (Q, \Sigma, \Gamma, \delta, q_0, q_F)$, kde:
\begin{itemize}
    \setlength{\itemsep}{0.05em}
    \item $Q$ je konečná množina {\normalfont vnitřních (řídicích) stavů},
    \item $\Sigma$ je konečná množina symbolů nazývaná {\normalfont vstupní abeceda}, $\Delta \not\in \Sigma$,
    \item $\Gamma$ je konečná množina symbolů, $\Sigma \subset \Gamma$, $\Delta \in \Gamma$, nazývaná {\normalfont pásková abeceda},
    \item $\delta : (Q\backslash\{q_F\})\!\times\!\Gamma \rightarrow Q\!\times\!(\Gamma\!\cup\!\{L, R\})$,\,kde\,$L,R \not\in \Gamma$, je parciální {\normalfont přechodová funkce}, a
    \item $q_0 \in Q$ je {\normalfont počáteční stav} a $q_f \in Q$ je {\normalfont koncový stav}.
\end{itemize}
\end{deft}

Symbol $\Delta$ značí tzv. \emph{blank} (prázdný symbol), který se vyskytuje na místech pásky, která nebyla ještě použita.

\emph{Konfigurace pásky} se skládá z~nekonečného řetězce, který reprezentuje obsah pásky a~pozice hlavy na tomto řetězci. Jedná se o~prvek množiny $\{\gamma\Delta^\omega\ |\ \gamma\in\Gamma^*\} \times \mathbb{N}$\footnote{Pro libovolnou abecedu $\Sigma$ je $\Sigma^\omega$ množina všech \emph{nekonečných} řetězců nad $\Sigma$, tj. nekonečných posloupností symbolů ze $\Sigma$.}. \emph{Konfiguraci pásky} obvykle zapisujeme jako $\Delta xyz\underline{z}x \Delta...$ (podtržení značí pozici hlavy). \emph{Konfigurace stroje} je pak dána stavem řízení a~konfigurací pásky. Formálně se jedná o~prvek množiny $Q \times \{\gamma\Delta^\omega\ |\ \gamma\in\Gamma^*\} \times \mathbb{N}$.

\subsection{Podsekce obsahující větu a odkaz}
\begin{deft}
\label{deft:retezec}
{\normalfont Řetězec $w$ nad abecedou $\Sigma$ je přijat TS $M$} jestliže $M$ při aktivaci z počáteční konfigurace pásky $\underline{\Delta}w\Delta...$ a počátečního stavu $q_0$ zastaví přechodem do koncového stavu $q_F$, tj. $(q_0, \Delta w \Delta^\omega,0)\ \mathop{\vdash}\limits_{M}^{*}\ (q_F,\gamma,n)$ pro nějaké $\gamma \in \Gamma^*$ a $n \in \mathbb{N}$.

Množinu $L(M) = \{w\ |\ w\ \text{je přijat TS}\ M\} \subseteq \Sigma^*$ nazýváme {\normalfont jazyk přijímaný TS} $M$.
\end{deft}

Nyní si vyzkoušíme sazbu vět a důkazů opět s použitím balíku \verb|amsthm|.
\begin{lem}
Třída jazyků, které jsou přijímány TS, odpovídá {\normalfont rekurzivně vyčíslitelným jazykům.}

\begin{proof}
V důkaze vyjdeme z Definice \ref{deft:turinguv_stroj} a \ref{deft:retezec}.
\end{proof}
\end{lem}

\section{Rovnice}
Složitější matematické formulace sázíme mimo plynulý text. Lze umístit několik výrazů na jeden řádek, ale pak je třeba tyto vhodně oddělit, například příkazem \verb|\quad|.

\[
\sqrt[i]{x^3_i}\quad\text{kde}\ x_i\ \text{je}\ i\text{-té sudé číslo}\quad y^{2\cdot y_i}_i \ne y^{y^{y_i}_i}_i
\]

V rovnici \eqref{eq:first} jsou využity tři typy závorek s různou explicitně definovanou velikostí.

\begin{eqnarray}
x &= &\bigg\{\Big(\big[a + b \big] * c \Big)^d \oplus 1 \bigg\}\label{eq:first}\\
y &= &\lim_{x\to\infty}\frac{\sin^2 x + \cos^2 x}{\frac{1}{\log_{10} x}}\label{eq:second}
\end{eqnarray}

V této větě vidíme, jak vypadá implicitní vysázení limity $\lim_{n\to\infty} f(n)$ v~normálním odstavci textu. Podobně je to i~s~dalšími symboly jako $\sum_{i=1}^{n} 2^i$ či $\bigcap_{A\in\mathcal{B}} A$. V~případě vzorců $\lim\limits_{n\to\infty} f(n)$ a $\sum\limits_{i=1}^{n} 2^i$ jsme si vynutili méně úspornou sazbu příkazem \verb|\limits|.

\section{Matice}
Pro sázení matic se velmi často používá prostředí \verb|array| a~závorky (\verb|\left|, \verb|\right|).

\[\left(
\begin{array}{ccc}
    a+b & \widehat{\xi + \omega}  & \hat{\pi}\\
    \vec{\mathbf{a}} & \overleftrightarrow{AC} & \beta
\end{array}\right) = 1 \Longleftrightarrow \mathbb{Q} = \mathcal{R}
\]
Prostředí \verb|array| lze úspěšně využít i~jinde.

\[\binom{n}{k} = \left\{
\begin{array}{c l}
0 & \text{pro } k < 0\ \text{nebo } k > n \\ % viz poznámka v hlavičce
\frac{n!}{k!(n-k)!} & \text{pro } 0 \leq k \leq n.
\end{array} \right.
\]

\end{document}
